\documentclass[12pt,a4paper]{article}
\usepackage[utf8]{inputenc}
\usepackage{geometry}
\usepackage{hyperref}
\usepackage{listings}
\usepackage{xcolor}

\geometry{margin=1in}

\definecolor{codegray}{rgb}{0.95,0.95,0.95}
\definecolor{keywordcolor}{rgb}{0.36,0.54,0.66}

\lstset{
    backgroundcolor=\color{codegray},
    basicstyle=\ttfamily,
    keywordstyle=\color{keywordcolor}\bfseries,
    frame=single,
    breaklines=true,
    postbreak=\mbox{\textcolor{red}{$\hookrightarrow$}\space}
}

\title{Early Care Gateway}
\author{Execution Instructions}
\date{}

\begin{document}

\maketitle

\tableofcontents
\newpage

\section{Prerequisites}
Before running the project, ensure the following tools are installed on your system:

\begin{itemize}
    \item \textbf{Python 3}: Generate the JWT secret key
    \item \textbf{Docker}: Build containers
    \item \textbf{Docker Compose}: Deploy multi-container system
    \item \textbf{Git}: Clone the repository
\end{itemize}

\noindent
Check versions to ensure installation:
\begin{lstlisting}[language=bash]
python --version
docker --version
docker-compose --version
git --version
\end{lstlisting}

\section{Getting Started (Step-by-Step)}

\subsection*{1. Clone the repository}
\begin{lstlisting}[language=bash]
git clone https://github.com/carrieroroberto/early_care_gateway.git
cd early_care_gateway
\end{lstlisting}

\subsection*{2. Configure environment variables}
The file \texttt{.env.example} is only a template of the environment configuration used by the system (original \texttt{.env} file should not be shared as a security best practice).  

You should either rename it to \texttt{.env} or create a new \texttt{.env} file with the correct values for your setup.

\subsubsection*{Generate a secret key}
If you have Python installed locally, you can generate a secure secret key with the following command:

\begin{lstlisting}[language=bash]
python secret_key_generator.py
\end{lstlisting}
Then, copy the generated key into your \texttt{.env} file under \texttt{SECRET\_KEY}.

\subsubsection*{Example .env configuration}
Then, the \texttt{.env} file should then appear as follows:

\begin{lstlisting}
POSTGRES_HOST=postgres
POSTGRES_USER=user
POSTGRES_PASSWORD=password
POSTGRES_DB=postgres
POSTGRES_PORT=5432

PGADMIN_EMAIL=admin@admin.com
PGADMIN_PASSWORD=admin

SECRET_KEY=123456789abcdefgh
\end{lstlisting}

\subsection*{3. Run the system}
Start the project using Docker Compose:

\begin{lstlisting}[language=bash]
docker-compose up --build
\end{lstlisting}

\noindent
This command will:
\begin{itemize}
    \item Build all Docker images (Backend, Frontend, Databases)
    \item Start the containers
    \item Link services for communication across networks and volumes
\end{itemize}

\subsection*{4. Use the system}
Browse the Web Application user interface at http://localhost:3000 to try the system, or use an API client (e.g. Postman) to test the API by following the related documentation.

\section{Troubleshooting}
\begin{itemize}
    \item Monitor the logs in the terminal to ensure that all services have been initialized without errors.
    \item Ensure that no other services are running on the ports used by the system to avoid conflicts.
    \item If Docker images fail to build, try stopping and removing existing containers with \texttt{docker-compose down}, then rebuild with \texttt{docker-compose up --build}.
    \item On Linux systems, Docker commands may require \texttt{sudo}.
    \item Access pgAdmin at \texttt{http://localhost:8080} using the credentials defined in your \texttt{.env} file to manage the database in a user-friendly interface.
\end{itemize}

\end{document}