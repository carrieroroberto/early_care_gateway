\documentclass[12pt,a4paper]{article}
\usepackage[utf8]{inputenc}
\usepackage{geometry}
\usepackage{hyperref}
\usepackage{listings}
\usepackage{xcolor}

\geometry{margin=1in}

\definecolor{codegray}{rgb}{0.95,0.95,0.95}
\definecolor{keywordcolor}{rgb}{0.36,0.54,0.66}

\lstset{
    backgroundcolor=\color{codegray},
    basicstyle=\ttfamily,
    keywordstyle=\color{keywordcolor}\bfseries,
    frame=single,
    breaklines=true,
    postbreak=\mbox{\textcolor{red}{$\hookrightarrow$}\space}
}

\title{Early Care Gateway}
\author{Execution Instructions}
\date{}

\begin{document}

\maketitle

\tableofcontents
\newpage

\section{Prerequisites}
Before running the project, ensure the following tools are installed on your system:

\begin{itemize}
    \item \textbf{Python 3} (for generating secret keys)
    \item \textbf{Docker} (for containerized deployment)
    \item \textbf{Docker Compose} (for orchestrating multi-container setup)
    \item \textbf{Git} (for cloning the repository)
\end{itemize}

\noindent
Check versions to ensure installation:
\begin{lstlisting}[language=bash]
python --version
docker --version
docker-compose --version
git --version
\end{lstlisting}

\section{Quickstart}

\subsection{1. Clone the repository}
\begin{lstlisting}[language=bash]
git clone https://github.com/carrieroroberto/early_care_gateway.git
cd early_care_gateway
\end{lstlisting}

\subsection{2. Configure environment variables}
The file \texttt{.env.example} is only a template of the environment configuration used by the system (original \texttt{.env} file should not be shared as a security best practice).  

You should either rename it to \texttt{.env} or create a new \texttt{.env} file with the correct values for your setup.

\subsubsection{Generate a secret key}
If you have Python installed locally, you can generate a secure secret key with the following command:

\begin{lstlisting}[language=bash]
cd backend
python secret_key_generator.py
\end{lstlisting}
Then, copy the generated key into your \texttt{.env} file under \texttt{SECRET\_KEY}.

\subsubsection{Example .env configuration}
\begin{lstlisting}
POSTGRES_USER=postgres_user
POSTGRES_PASSWORD=postgres_password
POSTGRES_DB=db_name
POSTGRES_PORT=5432

PGADMIN_EMAIL=pgadmin_email
PGADMIN_PASSWORD=pgadmin_password

SECRET_KEY=generated_secret_key_here
\end{lstlisting}

\subsection{3. Run the project}
Navigate to the backend directory and start the project using Docker Compose:

\begin{lstlisting}[language=bash]
cd backend
docker-compose up --build
\end{lstlisting}

\noindent
This command will:
\begin{itemize}
    \item Build all Docker images (Backend, Frontend, Databases)
    \item Start the containers
    \item Link services for communication across networks and volumes
\end{itemize}

\subsection{4. Use the system}
\begin{itemize}
    \item Browse at http://localhost:port and start working on.
\end{itemize}

\section{Troubleshooting}
\begin{itemize}
    \item Monitor the logs in the terminal to ensure that all services have been initialized without errors.
    \item Ensure that no other services are running on the ports used by the system (e.g., PostgreSQL: 5432, pgAdmin: 8080) to avoid conflicts.
    \item If Docker images fail to build, try stopping and removing existing containers with \texttt{docker-compose down}, then rebuild with \texttt{docker-compose up --build}.
    \item On Linux systems, Docker commands may require \texttt{sudo}.
    \item Access pgAdmin at \texttt{http://localhost:8080} using the credentials defined in your \texttt{.env} file to manage the database in a user-friendly interface.
\end{itemize}

\end{document}