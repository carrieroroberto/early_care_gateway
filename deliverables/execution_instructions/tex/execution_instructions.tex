\documentclass[12pt,a4paper]{article}

\usepackage[utf8]{inputenc}
\usepackage{geometry}
\usepackage{listings}
\usepackage{xcolor}
\usepackage{parskip}

\geometry{margin=1in}

\definecolor{codegray}{rgb}{0.95,0.95,0.95}
\definecolor{keywordcolor}{rgb}{0.36,0.54,0.66}

\lstset{
    backgroundcolor=\color{codegray},
    basicstyle=\ttfamily,
    keywordstyle=\color{keywordcolor}\bfseries,
    frame=single,
    breaklines=true,
    postbreak=\mbox{\textcolor{red}{$\hookrightarrow$}\space}
}

\usepackage[hidelinks]{hyperref}

\title{Early Care Gateway}
\author{Execution Instructions}
\date{}

\begin{document}

\maketitle
\tableofcontents
\newpage

\section{Prerequisites}
Prior to running the project, ensure that your system meets the following software and hardware requirements:

\begin{itemize}
    \item \textbf{Python 3}: Generate the JWT secret key
    \item \textbf{Pip}: Install testing dependencies
    \item \textbf{Docker (and WSL for Windows)}: Build containers
    \item \textbf{Docker Compose V2}: Deploy the multi-container system
    \item \textbf{Git}: Clone the repository
    \item \textbf{At least 2 GB of Disk Space}: Host AI models and system components
    \item \textbf{Recommended at least 8 GB RAM}: Ensure stable execution
\end{itemize}

Verify that the tools are correctly installed:

\begin{lstlisting}[language=bash]
git --version
python --version
pip --version
wsl --version   # Windows only
docker --version
docker compose version
\end{lstlisting}

\section{Getting Started (Step-by-Step)}
\subsection*{1. Clone the Repository}
\addcontentsline{toc}{subsection}{1. Clone the Repository}

\begin{lstlisting}[language=bash]
git clone https://github.com/carrieroroberto/early_care_gateway
cd early_care_gateway
\end{lstlisting}

\subsection*{2. Configure Environment Variables}
\addcontentsline{toc}{subsection}{2. Configure Environment Variables}
The file \texttt{.env.example} serves as a template for the environment configuration required by the system.  
Rename it to \texttt{.env}, or create a new \texttt{.env} file and fill it with the correct values for your setup (e.g. database user and password).

\subsubsection*{Generate Keys}
If Python is installed on your machine, you can generate a secure SHA256 secret key using:

\begin{lstlisting}[language=bash]
python secret_key_generator.py
\end{lstlisting}

Copy the generated key and paste it into your \texttt{.env} file under \texttt{SECRET\_KEY}.

If you already have a Gemini API key, place it under \texttt{GOOGLE\_API\_KEY}.  
Otherwise, you can obtain a free API key by visiting  
\href{https://aistudio.google.com/app/api-keys}{Google AI Studio}.  
After logging in and creating a project, you will be able to generate your personal API key.

\subsubsection*{Example .env Configuration}
Your \texttt{.env} file should look similar to the following:

\begin{lstlisting}
# PostgreSQL database connection settings
POSTGRES_HOST = postgres
POSTGRES_USER = earlycaregateway
POSTGRES_PASSWORD = your_password
POSTGRES_DB = earlycaregateway
POSTGRES_PORT = 5432

# pgAdmin credentials
PGADMIN_EMAIL = admin@earlycaregateway.com
PGADMIN_PASSWORD = your_password

# Application secret key
SECRET_KEY = your_secret_key

# External API keys
GOOGLE_API_KEY = your_api_key

# Microservices URLs (internal Docker network addresses)
GATEWAY_URL = http://gateway:8000/gateway
AUTHENTICATION_URL = http://auth_service:8000/authentication
DATA_PROCESSING_URL = http://data_service:8000/data_processing
EXPLAINABLE_AI_URL = http://xai_service:8000/explainable_ai
AUDIT_URL = http://audit_service:8000/audit

# Frontend application settings
REACT_APP_API_URL = http://localhost:8002/gateway
\end{lstlisting}

\subsection*{3. Run the System}
\addcontentsline{toc}{subsection}{3. Run the System}
You can start the system using \texttt{the run.bat} (Windows) or \texttt{run.sh} (macOS/Linux) scripts, that automatically launch the services and run automated integration tests.

Alternatively, you may manually start the system with:

\begin{lstlisting}[language=bash]
docker compose up --build
\end{lstlisting}

This command will:
\begin{itemize}
    \item Build all Docker images (Backend, Frontend, Database)
    \item Start the containers
    \item Connect services through networks and volumes
\end{itemize}

\subsection*{4. Use the System}
\addcontentsline{toc}{subsection}{4. Use the System}
Access the web application at:

\begin{center}
\texttt{http://localhost:3000}
\end{center}

You may also use an API client, such as Postman, to interact directly with the API endpoints following the provided documentation.

\section{Troubleshooting}

\begin{itemize}
    \item Monitor logs to verify that all services start correctly:  
\begin{lstlisting}[language=bash]
docker compose logs -f
\end{lstlisting}
    \item Ensure no other applications are using ports required by the system.
    \item If Docker images fail to build, try stopping and removing existing containers and volumes using:  
\begin{lstlisting}[language=bash]
docker compose down -v
\end{lstlisting}
    \item On Linux systems, Docker commands may require \texttt{sudo}.
    \item Access pgAdmin at \texttt{http://localhost:8080} using the credentials provided in your \texttt{.env} file.
\end{itemize}

\end{document}