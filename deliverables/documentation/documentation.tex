\documentclass[12pt,a4paper]{article}
\usepackage[utf8]{inputenc}
\usepackage{geometry}
\usepackage{longtable}
\usepackage{hyperref}
\usepackage{listings}
\usepackage{xcolor}

\geometry{margin=1in}

\definecolor{jsonbg}{rgb}{0.95,0.95,0.95}
\lstdefinelanguage{json}{
    basicstyle=\ttfamily\footnotesize,
    numbers=left,
    numberstyle=\tiny,
    stepnumber=1,
    numbersep=5pt,
    showstringspaces=false,
    breaklines=true,
    frame=single,
    backgroundcolor=\color{jsonbg}
}

\definecolor{sqlbg}{rgb}{0.97,0.97,0.97}

\lstdefinelanguage{SQL}{
    morekeywords={
        SELECT, FROM, WHERE, AND, OR, INSERT, INTO, VALUES,
        UPDATE, DELETE, CREATE, TABLE, IF, NOT, EXISTS,
        PRIMARY, KEY, DEFAULT, CURRENT_TIMESTAMP, INTEGER,
        VARCHAR, SERIAL, TIMESTAMP
    },
    sensitive=false,
    morecomment=[l]{--},
    morestring=[b]',
}

\lstdefinestyle{sqlstyle}{
    language=SQL,
    basicstyle=\ttfamily\footnotesize,
    keywordstyle=\bfseries,
    stringstyle=,
    commentstyle=,
    backgroundcolor=\color{sqlbg},
    numbers=left,
    numberstyle=\tiny,
    stepnumber=1,
    frame=single,
    breaklines=true,
    showstringspaces=false
}

\title{EarlyCare Gateway}
\author{API Documentation}
\date{}

\begin{document}

\maketitle

\tableofcontents
\newpage

\section{Introduction}
This documentation describes the EarlyCare Gateway system. It follows a microservices architecture, with a Gateway, an Authentication Service, a Data Processing Service, an Explainable AI Service, and an Audit Service.

Communication between services uses JSON format for requests and responses. Authentication is handled with JSON Web Tokens (JWT), and error handling uses standard HTTP status codes with JSON payloads.

\subsection{Dynamic Documentation and Testing}
FastAPI automatically generates interactive API documentation accessible at \texttt{/docs}. This interface allows to inspect all endpoints, view request and response schemas, and perform test calls directly from the browser, facilitating end-to-end verification.

\subsection{Status and Error Codes (RFC 7807)}
All the API endpoints illustrated in this documentation follow the RFC 7807 standard for error format and status codes, as shown in Table \ref{table-errors}.

\begin{longtable}{|l|p{10cm}|}
\hline
\textbf{Status} & \textbf{Description} \\
\hline
200 OK & Success \\
201 Created & Resource successfully created \\
400 Bad Request & Invalid input, missing parameters \\
401 Unauthorized & Invalid/expired JWT or credentials \\
404 Not Found & Resource not found \\
500 Internal Server Error & Unexpected server error \\
\hline
\caption{Status and Error Codes Overview}
\label{table-errors}
\end{longtable}

\section{User Endpoints}
\subsection{POST /register}
\begin{itemize}
    \item \textbf{Purpose:} Register a new doctor
    \item \textbf{Authentication:} None
    \item \textbf{Request Body:}
\begin{lstlisting}[language=json]
{
  "name": "Mario",
  "surname": "Rossi",
  "email": "mario.rossi@email.com",
  "password": "RossiMarioPassword25"
}
\end{lstlisting}
    \item \textbf{Response Body (success):}
\begin{lstlisting}[language=json]
{
  "message": "Doctor registered successfully"
}
\end{lstlisting}
    \item \textbf{Response Body (error):}
\begin{lstlisting}[language=json]
{
  "error": "Email already in use"
}
\end{lstlisting}
    \item \textbf{Status Codes:}
    \begin{itemize}
        \item 201 Created -- registration successful
        \item 400 Bad Request -- invalid input or email already exists
    \end{itemize}
    \item \textbf{Flow Notes:}
    \begin{itemize}
        \item Gateway forwards to Authentication Service
        \item Authentication Service hashes password and creates unique \texttt{doctor\_id}
        \item Asynchronous notification sent to Audit Service
    \end{itemize}
\end{itemize}

\subsection{POST /login}
\begin{itemize}
    \item \textbf{Purpose:} Authenticate doctor and issue JWT
    \item \textbf{Authentication:} None
    \item \textbf{Request Body:}
\begin{lstlisting}[language=json]
{
  "email": "mario.rossi@email.com",
  "password": "RossiMarioPassword25"
}
\end{lstlisting}
    \item \textbf{Response Body (success):}
\begin{lstlisting}[language=json]
{
  "token": "jwt_token"
}
\end{lstlisting}
    \item \textbf{Response Body (error):}
\begin{lstlisting}[language=json]
{
  "error": "Invalid credentials"
}
\end{lstlisting}
    \item \textbf{Status Codes:}
    \begin{itemize}
        \item 200 OK -- login successful
        \item 401 Unauthorized -- invalid credentials
    \end{itemize}
    \item \textbf{Flow Notes:}
    \begin{itemize}
        \item JWT contains \texttt{doctor\_id} in payload, signed with secret
        \item Gateway forwards token to UI for subsequent requests
        \item Audit Service notified asynchronously
    \end{itemize}
\end{itemize}

\subsection{POST /analyse}
\begin{itemize}
    \item \textbf{Purpose:} Execute diagnosis
    \item \textbf{Authentication:} JWT required (\texttt{Authorization: Bearer <JWT>})
    \item \textbf{Request Body:}
\begin{lstlisting}[language=json]
{
  "data": image | object | "text"
}
\end{lstlisting}
    \item \textbf{Response Body (success):}
\begin{lstlisting}[language=json]
{
  "report_id": 192,
  "diagnosis": "Diabetes Type 2",
  "explanation": "High glucose levels detected..."
}
\end{lstlisting}
    \item \textbf{Response Body (error):}
\begin{lstlisting}[language=json]
{
  "error": "Invalid token"
}
\end{lstlisting}
    \item \textbf{Status Codes:}
    \begin{itemize}
        \item 200 OK -- analysis completed
        \item 400 Bad Request -- invalid/missing data
        \item 401 Unauthorized -- invalid or expired JWT
    \end{itemize}
    \item \textbf{Flow Notes:}
    \begin{itemize}
        \item Gateway validates JWT via Authentication Service
        \item Gateway sends data to Data Processing Service
        \item Data Processing Service saves processed data in its database and returns \texttt{processed\_data\_id}
        \item Explainable AI retrieves data and executes model
        \item Diagnosis saved in reports database
        \item Audit Service notified asynchronously
    \end{itemize}
\end{itemize}

\subsection{GET /reports}
\begin{itemize}
    \item \textbf{Purpose:} Retrieve all reports for the doctor, optionally filtered by patient
    \item \textbf{Authentication:} JWT required (\texttt{Authorization: Bearer <JWT>})
    \item \textbf{Query Parameters (optional):} \texttt{patient\_cf}
    \item \textbf{Response Body:}
\begin{lstlisting}[language=json]
[
  {
    "report_id": 192,
    "patient_hashed_cf": ABCDEF12G34H567I",
    "diagnosis": "Diabetes Type 2",
    "created_at": "2025-11-14 10:00:00"
  }
]
\end{lstlisting}
    \item \textbf{Status Codes:}
    \begin{itemize}
        \item 200 OK -- reports retrieved
        \item 401 Unauthorized -- invalid JWT
        \item 404 Not Found -- no reports found
    \end{itemize}
    \item \textbf{Flow Notes:}
    \begin{itemize}
        \item Gateway validates JWT and extracts \texttt{doctor\_id}
        \item Calls Explainable AI to query reports database with optional \texttt{hashed\_patient\_cf}
    \end{itemize}
\end{itemize}

\section{Internal Endpoints}
Authentication Service exposes POST /validate to verify JWT and /register, /login endpoints for login and registration.
Data Processing Service exposes POST /process to process raw data and GET /data/{processed-data-id} to retrieve processed data.
Explainable AI Service exposes POST /analyse for performing diagnosis and GET /reports[?patient\_id=id] to retrieve reports.
Audit Service exposes POST /log to record events from other services.

\subsection{Authentication Service}
\subsubsection*{POST authentication/register}
This internal endpoint is invoked by the Gateway when a registration request is received. It creates a new doctor account, hashes the password, stores the record in the \texttt{doctors} table, and logs the operation.

Example request:
\begin{lstlisting}[language=json]
{
  "name": "Mario",
  "surname": "Rossi",
  "email": "mario.rossi@email.com",
  "password": "RossiMarioPassword25"
}
\end{lstlisting}

Example response:
\begin{lstlisting}[language=json]
{
  "message": "Doctor registered successfully"
  "doctor_id": 1,
}
\end{lstlisting}

If the email already exists, the service returns a 400 status with an error message.

\subsubsection*{POST authentication/login}
This endpoint validates credentials during login. If the email and password match an existing record, the service generates and returns a JWT.

Example request:
\begin{lstlisting}[language=json]
{
  "email": "mario.rossi@mail.com",
  "password": "RossiMarioPassword25"
}
\end{lstlisting}

Example successful response:
\begin{lstlisting}[language=json]
{
  "message": "Doctor logged in successfully"
  "token": "jwt_token"
}
\end{lstlisting}

If authentication fails, it returns a 401 error with a corresponding message, and the event is logged.

\subsubsection*{POST authentication/validate}
This endpoint verifies the validity of a JWT sent by the Gateway. The request must include the token in JSON form. If the token is valid, the service returns the associated doctor identifier. Otherwise, it returns an error.

Example request:
\begin{lstlisting}[language=json]
{
  "token": "jwt_token"
}
\end{lstlisting}

Example successful response:
\begin{lstlisting}[language=json]
{
  "message": "Token validated successfully"
  "doctor_id": id
}
\end{lstlisting}

Example error response:
\begin{lstlisting}[language=json]
{
  "error": "Invalid or expired token"
}
\end{lstlisting}

Status codes include 200 for valid tokens, 400 for missing or malformed tokens, and 401 for invalid or expired tokens. After every validation attempt, the Authentication Service notifies the Audit Service about the outcome to ensure traceability.

\subsection{Audit Service}
The Audit Service is responsible for recording and retrieving logs from all microservices. Its endpoints follow a structured JSON format and operate asynchronously relative to the business logic of other services. It implements an Observer Pattern that decouples monitoring concerns from core functionalities.

\subsubsection*{POST /audit/log}
This endpoint receives log entries from other services. Each log includes the service name, the type of event, a description, and optional identifiers related to doctors, processed data, or reports.

Example request:
\begin{lstlisting}[language=json]
{
  "service": "authentication",
  "event": "register_success",
  "description": "Doctor registered successfully"
}
\end{lstlisting}

Example response:
\begin{lstlisting}[language=json]
{
  "message": "Log created successfully",
  "log_id": 100,
  "created_at": "2025-11-25 12:00:00"
}
\end{lstlisting}

Typical status codes include 201 when logs are successfully written and 400 for malformed payloads.

\subsubsection*{GET /audit/logs}
This endpoint allows internal retrieval of all stored logs. It supports optional query parameters such as \texttt{service}, \texttt{event}, \texttt{doctor\_id} or \texttt{patient\_hashed\_cf} to filter results.

Example response:
\begin{lstlisting}[language=json]
[
  {
    "log_id": 1,
    "created_at": "2025-11-14 10:00:00",
    "service": "authentication",
    "event": "login_success",
    "description": "Doctor logged in successfully",
    "doctor_id": 101
  },
  {
    "log_id": 2,
    "created_at": "2025-11-14 10:02:00",
    "service": "data_processing",
    "event": "data_processed",
    "description": "processed data stored in the database",
    "doctor_id": 101,
    "data_id": 44
  }
]
\end{lstlisting}

If no logs match the query, the service returns an empty array with status 200. In case of internal errors, a 500 error is produced with a diagnostic message.

\section{Databases}
The system uses a SQL database for structured data and a NoSQL database for unstructured data.

\subsection*{PostgreSQL}
The used SQL database is PostgreSQL. For simplicity there is a single physical instance of this database, but logically each microservice operates on separate schemas. This design preserves microservice isolation while allowing a simplified physical database for testing. Each service interacts only with its own table.

The Authentication Service uses the \texttt{doctors} table to store user information including name, surname, email, and hashed password. The Audit Service uses the \texttt{logs} table to store events such as registration, login attempts, data processing, and report access.

Example schema:
\begin{lstlisting}[style=sqlstyle]
CREATE TABLE IF NOT EXISTS doctors (
    id SERIAL PRIMARY KEY,
    name VARCHAR(50) NOT NULL,
    surname VARCHAR(50) NOT NULL,
    email VARCHAR(100) UNIQUE NOT NULL,
    hashed_password VARCHAR(255) NOT NULL
    );

CREATE TABLE logs (
    id SERIAL PRIMARY KEY,
    created_at TIMESTAMP DEFAULT CURRENT_TIMESTAMP,
    service VARCHAR(50) NOT NULL,
    event VARCHAR(50) NOT NULL,
    description VARCHAR(100) NOT NULL,
    doctor_id INTEGER,
    patient_hashed_cf VARCHAR(255),
    data_id INTEGER,
    report_id INTEGER
    );
\end{lstlisting}

\subsection{MongoDB}

\end{document}