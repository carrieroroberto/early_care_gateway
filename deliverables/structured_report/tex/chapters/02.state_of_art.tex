\chapter{State of the Art}

This chapter provides a comprehensive overview of the current methodologies employed in clinical triage and preliminary diagnosis. It analyses the evolution from standardized manual protocols to advanced data-driven solutions, exploring the potential for the adoption of Artificial Intelligence in clinical context.

\section{Current Triage Protocols}
Among the technologies and protocols currently used to manage patient flow, the Manchester Triage System~\cite{zachariasse2017validity} is one of the most globally adopted frameworks. Functionally, it gives the nurses the possibility to assign a clinical priority to patients based on presenting symptoms, allocating them to one out of five categories depending on the urgency, with the aim of reducing queues in emergency facilities.

However, according to a recent review by Andika et al.~\cite{andika2025effectiveness}, despite the system being widely adopted, it presents some limitations. This study states that measurement errors are strictly linked to personal factors of the healthcare staff, leading to variability in decision-making. Moreover, this article states that the issues of sub-triage (underestimating severity) and super-triage (overestimating severity) persist, and for these reasons there's the need to implement an automated system that aims to mitigate human error and improve classification accuracy.

\section{The Role of Artificial Intelligence} 
One of the possible tools to address limitations of manual protocols is Artificial Intelligence, mainly divided into Traditional Machine Learning and Large Language Models. Historically, AI models such as Random Forest and Gradient Boosting have proven highly effective when applied in this field, allowing to reach clinical benefits through structured data analysis. A systematic review by Sanchez et al.~\cite{sanchez2022machine} analysed many machine learning methods applied to triage, concluding that these algorithms demonstrate consistency of accuracy, sensitivity and specificity when compared to traditional tools.

However, the study highlights a critical limitation. Infact, while these algorithms perform really well with structured data, they struggle with unstructured data. To address this problem, the studies has shifted towards Large Language Models, commonly known as LLMs. While traditional algorithms rely on numerical inputs, these new implementations have the capability to process and interpret unstructured data, such as clinical notes and patient narratives.

As described by Thirunavukarasu et al.~\cite{thirunavukarasu2023large}, these models are really helpful. Unlike traditional classifiers trained on specific tasks, LLMs are able to synthesize and simulate clinical reasoning. Another really significative strength of these models is their multimodal architecture, allowing them to integrate text, images and signals.

\section{Explainability and Trust in Clinical AI} 
Despite the technical potential, the adoption of LLMs is not trusted by most medical professionals as they avoid to trust algorithms that provide diagnoses without an explanation. To overcome this problem, the integration of Explainable AI techniques has become a must have for modern CDSS. According to a review by Loh et al.~\cite{loh2022application}, implementing explainability methods is essential to give trust and transparency to medical staff. Techniques such as SHAP (SHapley Additive exPlanations) and LIME (Local Interpretable Model-agnostic Explanations) allow the system to highlight which features influenced the model's decision, providing the doctors with the rationale behind the AI's output.

In conclusion, the current landscape needs a system that combines reliability of established medical protocols with the reasoning power of modern AI algorithms. So, the project is designed to cover all the aspect treated in this section within a secure and explainable architecture.